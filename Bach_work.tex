\documentclass[12pt,a4paper,oneside,draft,titlepage]{article}
\usepackage{amsmath,amsthm,amssymb}
\usepackage{mathtext}
\usepackage[T1,T2A]{fontenc}
\usepackage[utf8]{inputenc}
\usepackage[english,bulgarian,ukrainian,russian]{babel}

\begin{document}

\tableofcontents

\part{Введение}

\part{Классификация массивных бетонных плотин}

\part{Анализ существующих контрфорсных плотин}

\part{Исходные данные для постановки исследований}

\part{Назначения параметров контрфорсной плотины n и B из условий требований устойчивости и прочности}

Гравитационные плотины сопротивляются нагрузкам за счет собственной прочности и прочности контакта с основанием. Прочность достигается подбором такой формы конструкции секции, чтобы в ней отсутствовали растягивающие напряжения, а также зоны, в которых касательные напряжения достигают предельных значений.

\section{Геометрические характеристики расчетного сечения}

Для выполнения статического расчета необходимого найти центр тяжести расчетного сечения.
Положение центра тяжести расчетного сечения:

$$ x_A = (0.5 * B ^ 2  d + a  b ^ 2 + a ^ 2  (b + a  \sqrt{1 + n ^ 2} / 3)  \sqrt{1 + n ^ 2}) / F $$

$$ x_B = B - x_A $$

Момент инерции расчетного сечения в главной центральной системе координат (формула для простоты разделена на 6 слагаемых):

$$ Jy_0 = B ^ 3  d / 12 $$

$$ Jy_1 = a  b ^ 3 / 6 $$

$$ Jy_2 = \frac {a ^ 4}{18} (1 + n ^ 2) ^ {1.5} $$

$$ Jy_3 = (x_A - 0.5  B) ^ 2  B  d $$

$$ Jy_4 = 2  (x_A - 0.5  b) ^ 2  a  b $$

$$ Jy_5 = a ^ 2 (x_A - b - a  \sqrt{1 + n ^ 2} / 3) ^ 2  \sqrt{1 + n ^ 2} $$

$$ Jy = \sum J_{yi} $$

Моменты сопротивления

$$W_A = \frac{J_y}{|x_A|}$$

$$W_B = \frac{J_y}{|x_B|}$$

$$F = B d + a (2  b + a  \sqrt{1 + n ^ 2})$$

$$a = (D - d) / 2 $$

\section{Нагрузки}

Список нагрузок, действующих на секцию плотины имеет следующий вид:

\begin{itemize}
	\item Собственный вес плотины
	\item Гидростатическая нагрузка
	\item Противодавление
	\item Давление взвешенных наносов
	\item Волновая нагрузка
\end{itemize}

Необходимо найти сумму сил от этих нагрузок и их моментов, затем используя критерий отсутствия растягивающих напряжений и критерий устойчивости плотины найти оптимальные параметры профиля плотины - ширину по основанию и уклон верховой грани.

Для простоты большие математические выражения были разделены на составные части.

\subsection{Нагрузка от собственного веса плотины}

При определении собственного веса плотины самым простым и понятным путем будет разделение бокового профиля сегмента плотины на элементы.

Вес элемента контрфорса с треугольным профилем:

$$ G_1 = 0.5  \gamma_c  B  H_t  d $$
где $\gamma_c$ - удельный вес бетона.

Момент веса этого элемента относительно главной центральной оси y:

$$ M_1 = \gamma_c  H_t  B  d  (B - 3  x_A + n  H_t) / 6 $$

Вес элементов оголовка контрфорса:

$$ G_2 = \gamma_c  H_t  (a  \sqrt{1 + n ^ 2} + 2  b)  a $$

Момент веса этого элемента относительно главной центральной оси y (выражение состоит из нескольких частей):

$$ M_{2,1} = b  (2  x_A - b - n  H_t) $$
 
$$ M_{2,2} = a  \sqrt{1 + n ^ 2}  (x_A - b - (n  H_t) / 2 - (a  \sqrt{1 + n ^ 2}) / 3) $$
 
$$ M_2 = -\gamma_c  H_t  a ( M_{2,1} + M_{2,2} ) $$


Вес элемента гребневого оголовка с прямоугольным профилем и его момент относительно главной центральной оси y:

$$ G_3 = \gamma_c  B_c  D  d_c $$
$$ M_3 = -G_3 * (x_A - n  H_t - e_c) $$
где: $ d_c = hw + d_{hw} + 0.6 $ - возвышение гребня плотины над уровнем верхнего бьефа,\\
$ d_{hw} = \pi  h_w ^ 2 / L_w $ - высота наката волны,
$B_c$ - ширина гребня плотины.

Вес элемента гребневого оголовка с треугольным профилем и его момент относительно главной центральной оси y: 

$$ G4 = 0.5  \gamma_c  D  a_1 ^ 2 / n $$
$$ M4 = -G4  (x_A - n  H_t + 2  a_1 / 3) $$
где: $ a_1 = (0.5  B_c - e_c) $, $ a_2 = (0.5  B_c + e_c) $\\

Вес изымаемого элемента треугольного сечения:

$$ G5 = -0.5  \gamma_c  n  D  a_2 ^ 2 / m ^ 2 $$

Момент веса этого элемента относительно главной центральной оси y (выражение состоит из нескольких частей):
$$ M_{5,1} = \frac{2  n  a_2 }{ 3  m} $$
$$ M_5 = -G_5 * (x_A - b - n  H_t + M_{5,1}) $$

Для нахождения веса изымаемого элемента трапецеидального сечения $G_6$ необходимо найти следующие величины:

$$ a^* = \frac{a_2  (b + a  \sqrt{1 + n ^ 2})  \sqrt{1 + n ^ 2}} {m  b} $$
$$ b^* = x_A - b - n  H_t - a  \sqrt{1 + n ^ 2} $$
$$ n_1 = \frac {1 - n  m}{n + m} $$

Вес изымаемого элемента трапецеидального сечения (выражение состоит из нескольких частей):

$$ G_{6,1} = a^* - \frac{2  a  \sqrt{1 + n ^ 2}  H_t} { 3  B} $$
 
$$ G_6 = -\gamma_c  a ^ 2  G_{6,1} $$

Момент веса этого фрагмента (выражение состоит из нескольких частей):

$$ M_{6,1} = \frac{n  (2 + n ^ 2)}{\sqrt{1 + n ^ 2}} + \frac{n_1  (2 - n ^ 2)}{ \sqrt{1 + n_1 ^ 2}} $$

$$ M_{6,2} = n  n_1  a^* + (n + n_1)  b^* + \frac{3  a  M_{6,1}}{8} $$

$$ M_{6,2,2} = b^* + \frac{n  a}{2} + \frac{2  a} { 3  \sqrt{1 + n ^ 2}} $$

$$ M_{6,3} = a^*  M_{6,2,2} - 2  a  M_{6,2} / 3 $$

$$ M_6 = \gamma_c  a ^ 2  M_{6,3} $$

\subsection{Гидростатическая нагрузка}

Гидростатическую нагрузку в данной работе предлагается разделить на вертикальную и горизонтальную составляющую.

Нагрузка со стороны верхнего бьефа, горизонтальная составляющая:

$$ W_{up,hrz} = 0.5  \gamma_w  H_t ^ 2  D $$
$$ M_{up,hrz} = \gamma_w  H_t ^ 3  D / 6 $$
где $\gamma_w$ - удельный вес воды.

Вертикальная составляющая:

$$ W_{up,vrt} = n  W_{up,hrz} $$
$$ M_{up,vrt} = -\gamma_w  n  D  H_t ^ 2  (3  x_A - n  H_t) / 6 $$

Нагрузка со стороны нижнего бьефа, горизонтальная составляющая:

$$ W_{low,hrz} = -0.5  \gamma_w  H_2 ^ 2  D $$
$$ M_{low,hrz} = W_{low,hrz} * H_2 / 3 $$
где $H_2$ - глубина воды в нижнем бьефе.

Вертикальная составляющая:

$$ W_{low,vrt} = (n - \frac{B  d}{H_t  D}) * W_{low,hrz} $$
 
Выражения для момента состоит из нескольких частей:

$$ M_{low,vrt,1} = |x_A| - b - \frac{a}{2} - \frac{n  H_2}{3} $$
$$ M_{low,vrt,2} = x_B - m  H_2 $$
$$ M_{low,vrt,3} = 2  n  a  M_{low,vrt,1} + m  M_{low,vrt,2}  d $$
$$ M_{low,vrt} = \frac{M_{low,vrt,3}  W_{low,vrt}}{D} $$

\subsection{Противодавление}
Нагрузка от противодавления действует на подошву плотины.

Сила взвешивания:
$$ U_1 = -\gamma_w  H_2  [D  b + \frac{a  (D + d) }{2} + (B - b - a) d] $$
$$ M_{U1} = 0 $$

Фильрационная составляющая, передающаяся на подошву плотины через бетонную подушку:
$$ U_2 = -\gamma_w (H_t - H_2)  (b + a)  D $$

Момент этой силы:
$$ M_{U2} = -U_2  (|x_A| - \frac{b + a}{2}) $$

\subsection{Давление наносов}

Данную нагрузку также предлагается разбить на горизонтальную и вертикальную составляющие.

Горизонтальная составляющая:
$$ W_{allu,hrz} = 0.5  \gamma_{allu}  H_{allu} ^ 2  D $$
Момент:
$$ M_{allu,hrz} = W_{allu,hrz}  H_{allu} / 3 $$
где $\gamma_{allu}$ - удельный вес взвешенных уплотнившихся наносов,\\
$H_{allu}$ - толщина слоя наносов,\\

Вертикальная составляющая:
$$ W_{allu,vrt} = 0.5  \gamma_{allu}  n  H_{allu} ^ 2  D $$
Момент:
$$ M_{allu,vrt} = -W_{allu,vrt}(|x_A| - \frac{n  H_{allu}}{3}) $$

\subsection{Волновая нагрузка}

Волновая нагрузка, очевидно, действует только в то время, когда поверхность водохранилища не покрыта льдом, за исключением теплых районов.

Горизонтальная составляющая волновой нагрузки:
$$ W_{wave,hrz} = \gamma_w  h_w  (\frac{L_w}{\pi}  + \frac{h_w}{2}) \frac{D}{2} $$
где $L_w$ - длина волны,\\
Момент этой нагрузки:
$$ M_{wave,hrz} = W_{wave,hrz}  (H_t - \frac{L_w }{ 2  \pi} + \frac{3 h_w}{8}) $$

Вертикальная составляющая:
$$ W_{wave,vrt} = W_{wave,hrz}  n $$

Моментом вертикальной составляющей можно пренебречь.

\subsection{Давление льда}

Поскольку в данной работе выбран теплый район строительства, следующие выражения не используются. При проектировании в холодных районах следует учесть эту составляющую при составлении сочетаний нагрузок.

$$ W_{ice} = R_{ice} h_{ice} D $$
$$ M_{ice} = W_{ice} (H_t - 0.45 h_{ice} $$

\subsection{Сочетания нагрузок}

Для строительного периода:
$$ N = G_1 + G_2 + G_3 + G_4 + G_5 + G_6 $$
$$ Q = 0 $$
$$ M = M_1 + M_2 + M_3 + M_4 + M_5 + M_6 $$

Для эксплуатационного периода:
$$ N = G_1 + G_2 + G_3 + G_4 + G_5 + G_6 + W_{up,vrt} + W_{low,vrt} + U_1 + U_2 + W_{wave,vrt} + W_{allu,vrt} $$
$$ Q = W_{up,hrz} + W_{low,hrz} + W_{allu,hrz} + W_{wave,hrz} $$
$$ M_{water} = M_{up,hrz} + M_{up,vrt} + M_{low,hrz} + M_{low,vrt} $$
$$ M = M_1 + M_2 + M_3 + M_4 + M_5 + M_6 + M_{water} + M_{U2} + M_{allu,hrz} + M_{allu,vrt} + M_{wave,hrz} $$

\section{Нахождение оптимальных параметров n и В численным методом}

Параметры n и B находятся из условий:
\begin{itemize}
	\item Отсутствие во всех точках растягивающих напряжений
	\item Устойчивость плотины сдвигающим усилиям
	\item При всех вышеперечисленных - минимальный расход бетона
\end{itemize}

Условие отсутствия растягивающих напряжений выражается уравнением:
$$ [\sigma_{z,A}]  B ^ 2 - N  B + 6  M = 0 $$

Параметр $[\sigma_{z,A}]$ - допускаемое напряжение в точке А, совмещающий в себе условие прочности и экономичности. Подбирать его следует, начиная с минимального значения 10 тонн на квадратный метр. Если этого оказывается недостаточно для обеспечения отсутствия растягивающих напряжений, можно увеличивать этот параметр, вплоть до 30.\newline

Условие устойчивости секции плотины имеет следующий вид:

$$ f  N + c  B - k_3  Q = 0 $$

где f,c - характеристики грунта основания,\\
$k_3$ - коэффициент запаса в зависимости от класса надежности сооружения. Для высоконапорных плотин (от 60 метров высотой) следует принять 1,25.

Для каждого значения n существует такое значение B, при котором выполняются оба условия.

В данной работе предлагается найти данные параметры численным методом.

\subsection{Общее описание метода}

Программа, предлагаемая в данной работе состоит из нескольких частей:

\begin{itemize}
	\item Блок импорта необходимых для работы библиотек,
	\item Функция расчета нагрузок, критериев устойчивости, прочности и построения эпюр, далее - функция статического расчета,
	\item Функция построения графика зависимости B(n) с учетом критериев прочности и устойчивости,
	\item Функция построения графиков зависимости параметров n и B от высоты плотины,
\end{itemize}

В программировании функции используются для тех участков кода, которые повторяются несколько раз с целью избежать их повторения. Функция может иметь свои внутренние переменные, выполнять над ними различные операции.

Помимо внутренних переменных в функцию можно передать переменные из основного блока программы, такие переменные называются параметрами функции.

Функция также может возвращать в основной блок программы переменные.

Функция статического расчета принимает на вход несколько параметров, самые важные из них - уклон верховой грани $n$, ширину секции плотины по основанию $B$, высоту треугольного профиля плотины $H_t$

Функция, используя выражения для вычисления нагрузок, может по усмотрению пользователя вернуть значения критериев устойчивости и отсутствия растягивающих напряжений, значения напряжений, а также проверить, не превышают ли касательные напряжения во всех точках предельных значений.

Критерии отсутствия растягивающих напряжений и устойчивости \\
принимают вид:

$$ С_{nostretch} = [\sigma_{z,A}]  B ^ 2 - N  B + 6  M $$

$$ C_{sustain} = f  N + c  B - k_3  Q $$


Компьютер перебирает все возможные значения n и B в заданном диапазоне.
Те значения n и B, при которых $С_{nostretch}$ и $C_{sustain}$ максимально близки к нулю записываются в массив.\\
На основе полученных значений строятся два графика, один для условия отсутствия растягивающих напряжений, другой для условия устойчивости.
Если все выполнено правильно, графики должны пересекаться в одной точке, которая и будет оптимальным решением.

\subsection{Функция статического расчета секции плотины}

Общий алгоритм работы функции:
\begin{itemize}
	\item Принять на вход в качестве параметров геометрические характеристики плотины: B, n, $H_t$, D, d, допускаемое напряжение $\sigma_z$ в точке А
	\item Найти значения всех нагрузок всех видов
	\item Собрать значения нагрузок в сочетания
	\item На основании сочетаний вычислить критерии отсутствия растягивающих напряжений и устойчивости
	\item Вернуть эти значения в основной блок программы
	\item Вычислить нормальные и касательные напряжения по всем точкам вдоль оси х
	\item Проверить, не превышают ли напряжения предельных значений
	\item Построить графики напряжений
\end{itemize}


\subsection{Функция построения графика зависимости\\ B от n}

\subsection{Функция построения графика зависимости B и n от высоты профиля плотины}

\part{Сопоставление касательных напряжений по подошве контрфорсной плотины с предельными значениями}


 Sigma Z
 $$ \sigma_z = \frac{N}{F} + \frac{M}{J_y}x $$
 $$ x \in [x_A, x_B] $$

 $$ \sigma_{z,A} = \frac{N}{F} + \frac{M}{J_y}x_A $$
 $$ \sigma_{z,B} = \frac{N}{F} + \frac{M}{J_y}x_B $$

 Sigma X
 $$ p_A = \gamma_w  H_t + \gamma_{allu}  H_{allu} $$
 $$ p_B = \gamma_w  H_2 $$

 $$ \sigma_{x,A} = (1 - n ^ 2)  p_A + n ^ 2  \sigma_{z,A} $$
 $$ \sigma_{x,B} = (1 - m ^ 2)  p_B + m ^ 2  \sigma_{z,B} $$

 $$ \sigma_x = \frac{\sigma_{x,A} (x_B - x) D + \sigma_{x,B} (x - x_A) d}{B [d(x)]} $$

 $$ [d(x)] =
 \begin{cases}
 D, \qquad \qquad \qquad \qquad \,x \in [x_A; x_{C1}]\\
 D - \frac{D - d}{x_{C2} - x_{C1}} (x - x_{C1}), \quad x \in [x_{C1}; x_{C2}]\\
 d, \qquad \qquad \qquad \qquad \; x \in [x_{C2}; x_B]\\
 \end{cases}
 $$

 Tau XZ
 
 $$ \tau_{xz,A} = n (p_A - \sigma_{z,A}) $$
 
 $$ \tau_{xz,B} = -m (p_B - \sigma_{z,B}) $$

 $$ \tau_{xz} = \tilde{\tau_{xz}} + \Delta \tau_{xz} $$

 $$ \tilde{\tau_{xz}} = \frac{\tau_{xz,A} (x_B - x) D + \tau_{xz,B} (x - x_A) d }{B [d(x)]} $$
 
 $$ x \in [x_A, x_B] $$

 $$ \Delta Q = Q - (\tau_{xz,A} D + \tau_{xz,B} d) B / 2 $$

 $$ \Delta \tau (x) =
 \begin{cases}
 0.5  (x_A ^ 2 - x ^ 2), \qquad \qquad \qquad \qquad \quad \: x \in [x_A; x_{C1}]\\
 \frac{3D(x_{C2}-x_{C1})(x_A ^ 2 - x ^ 2)+(D-d)(2x+x_{C1})(x-x_{C1})^2}{6[D(x_{C2}-x_{C1})-(D-d)(x-x_{C1})]}, \quad  x \in [x_{C1}; x_{C2}]\\
 \frac{3D(x_A^2-x^2)+(D-d)(3x^2-x_{C1}^2-x_{C1}x_{C2}-x_{C2}^2)}{6 d}, \quad \quad \: x \in [x_{C2}; 0]\\
 0.5  (x_B ^ 2 - x ^ 2), \qquad \qquad \qquad \qquad \quad \; x \in [0; x_B]\\
 \end{cases}
 $$

 $$ \Delta \tau_{xz} = \frac{\Delta Q}{J_y} \Delta \tau (x) $$

 $$ \tau_{xz,lim} = f  (\frac{\sigma_{z,A}  x_B - \sigma_{z,B}  x_A}{  B} + \frac{\sigma_{z,B} - \sigma_{z,A} } {B} x) + c $$
 
 $$ x \in [x_A, x_B] $$

 Sigma 1,2
 
$$ \sigma_{1,2} = \frac{\sigma_x + \sigma_z \pm \sqrt{(\sigma_x - \sigma_z) ^ 2 + 4  \tau_{xz} ^ 2}} {2} $$

\part{Рекомендации по назначению параметров контрфорсной плотины для различных её высот}

\end{document}